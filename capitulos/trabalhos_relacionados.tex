\chapter{Trabalhos Relacionados}\label{cap:trabalhos_relacionados}

A identificação de desinformação e a análise de posicionamentos têm sido amplamente investigadas na literatura devido à sua relevância para a compreensão dos fenômenos sociais, bem como pelo impacto cultural, político e econômico associado a esses temas. Nesta seção, são apresentados os principais trabalhos relacionados a essa temática, com destaque para suas contribuições, metodologias aplicadas e os pontos de melhoria que motivaram e influenciaram o presente estudo.
Inicialmente, discutimos pesquisas com escopo geral voltadas para a classificação de desinformação, cujo aprimoramento é um dos objetivos deste trabalho. Em seguida, apresentamos o estudo inicial que serviu como base para o desenvolvimento tanto do primeiro artigo quanto desta pesquisa, corroborando para a construção de uma base sólida para estes estudos.

O primeiro trabalho em questão, ~\cite{raphaelIC}, tem como foco a análise do posicionamento dos usuários em discussões \textit{online} sobre desinformação. A pesquisa investiga como os posicionamentos expressos em textos podem ser utilizados para identificar conteúdos potencialmente enganosos ou nocivos, com especial atenção às discussões sobre as urnas eletrônicas no Brasil, no período de fevereiro a novembro de 2022. Utilizando técnicas algorítmicas, o estudo aplica modelagem de tópicos e análise de interações nas redes sociais, com dados extraídos de postagens publicadas no \textit{Twitter} (atualmente \textit{X}) e conteúdo desinformativo verificado por ``agências'' de checagem de notícias. Tal extração textual foi auxiliada pelo processo manual na elaboração de termos-chave para obtenção de conteúdo com desinformação, sendo esta a principal lacuna que o presente estudo pretende preencher por meio da elaboração automatizada de termos-chave.

A pesquisa demonstrou a viabilidade da aplicação de técnicas de detecção de posicionamento e modelagem de tópicos para identificar desinformação e caracterizar o comportamento interacional dos usuários na propagação desse conteúdo, utilizando também técnicas de \textit{TF-IDF} para rotulação e análise \cite{raphaelIC}.

O segundo trabalho, ~\cite{automaticDetection}, tem como objetivo analisar os ataques ao sistema eleitoral brasileiro durante as eleições de 2022, especificamente no \textit{Twitter}. Adotando uma abordagem interdisciplinar e o uso de ferramentas computacionais de rotulação automatizada de perfis e análise de linguagem natural, o artigo identificou os principais tipos de discursos hostis e o posicionamento político dos perfis responsáveis por esses tipos de discursos contra as urnas eletrônicas, o Tribunal Superior Eleitoral (TSE) e os magistrados do tribunal. Os dados revelaram que perfis governistas, principalmente bolsonaristas, foram os responsáveis por uma maior produção de conteúdos hostis, incluindo xingamentos às urnas eletrônicas e ataques de ódio direcionados aos ministros do TSE.

Ademais, o estudo aplicou um método de detecção de posicionamento não supervisionado para agrupar os usuários em \textit{clusters} polarizados, com base nas contas que retuitaram. Essa técnica foi fundamentada em estudos que sugerem que os usuários tendem a polarizar suas opiniões e formar comunidades políticas, seguindo o princípio da homofilia. A análise revelou dois grandes \textit{clusters}: um representando usuários favoráveis à visão política dominante e outro, contrário. Esses \textit{clusters} mostraram-se densos em interações internas, com baixa proximidade entre si, caracterizando a polarização do debate. A abordagem permitiu entender melhor as dinâmicas de propagação de discursos tóxicos e a formação de ``bolhas'' ideológicas dentro da rede social, reforçando a relevância da polarização como um fator na disseminação de desinformação \cite{automaticDetection}.
Esses \textit{clusters} polarizados também foram utilizados para testar a extração de tuítes com base nos termos-chave obtidos nesta pesquisa, assim como foi feito com os termos-chave extraídos manualmente na primeira pesquisa citada. Portanto, esse artigo foi essencial para a comparação entre os métodos automáticos e manuais de extração de termos-chave que permitiu avaliar a eficácia da abordagem automatizada, oferecendo informações fundamentais sobre a viabilidade de sua aplicação em contextos de grande volume de dados.
